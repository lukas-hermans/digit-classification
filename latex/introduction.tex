Recently, the application of computer vision techniques has gained increasing importance in a broad variety of fields, e.g. for the development of autonomeous vehicles, for facial recognition, and for the detection of illnesses \cite{autonomeous2020, emotion2020, medicine2021}. This process is especially driven by the growing field of machine learning. \\

A common dataset to test and compare machine learning algorithms for image recognition as a subfield of computer vision is the MNIST database of handwritten digits. The dataset contains $60000$ training examples and $10000$ test examples of handwritten digits from $0$ to $9$. Originally, the dataset was published by Y. LeCun et al \cite{MNIST}. In this paper, in order to simplify the work with the images, a revision of the original dataset is adopted \cite{KaggleData}. In the following, the revised dataset is simply refered to as MNIST dataset. \\

The objective of the present work is to train two different approaches of the multiclass kernel perceptron algorithm using the training examples in the MNIST database. Their performance for different choices of hyperparameters is compared with the help of the test examples. The result are two implementations of the kernel perceptron for the recognition of handwritten digits, and their corresponding test errors. In the last step, their performance is compared with other approaches that can be found in the literature.

- structure of paper